% Dies ist Teil der Vorlesung Physik auf dem Computer, SS 2012, 2013
% Axel Arnold, Universitaet Stuttgart.
% 
% Dieses Werk ist unter einer Creative Commons-Lizenz vom Typ
% Namensnennung-Weitergabe unter gleichen Bedingungen 3.0 Deutschland
% zugänglich. Um eine Kopie dieser Lizenz einzusehen, konsultieren Sie
% http://creativecommons.org/licenses/by-sa/3.0/de/ oder wenden Sie sich
% schriftlich an Creative Commons, 444 Castro Street, Suite 900, Mountain
% View, California, 94041, USA.
\chapter{Lineare Algebra \textrm{I}}
\index{Gleichungssysteme>lineare}

Lineare Gleichungssysteme sind die einfachste Art von
Gleichungssystemen, für die sich zum Beispiel leicht bestimmen lässt,
ob und wie viele Lösungen es gibt. Daher führt man auch die Lösung
komplexerer Probleme, wie zum Beispiel Differentialgleichungen, oft
auf die Lösung eines Satzes von linearen Gleichungssystemen
zurück. Lineare Gleichungssysteme sind in diesem Sinne eine der
wesentlichen Grundlagen der numerischen Mathematik. Der händischen
Lösung der Systeme steht dabei vor allem ihre Größe im Weg ---
Finite-Elemente-Rechnungen können leicht die Lösung von
Gleichungssystemen mit Millionen von Variablen erfordern. Mit modernen
Algorithmen und Computern lassen sich solche Gleichungssysteme
allerdings schnell und zuverlässig lösen. In diesem Kapitel lernen wir
die grundlegende Methode zum Lösen von Gleichungssystemen kennen,
nämlich die allgemeine, aber langsame Gaußelimination. Daneben lernen
wir noch die LU-Zerlegung und die Choleskyzerlegung kennen, die mit
etwas Vorarbeit eine effizientere Lösung erlauben und im Folgenden oft
zum Einsatz kommen werden.

Wir betrachten also folgendes Problem: Sei $A=(a_{ik})\in\RR^{m\times
  n}$, $b\in\RR^m$. Gesucht ist die Lösung $x\in\RR^n$ des
Gleichungssystems
\begin{equation}
  \label{eq:lgs}
  \begin{matrix}
    a_{11}x_1 &+&  a_{12}x_2 &+& \ldots &+& a_{1n}x_n &=& b_1\\
    a_{21}x_1 &+&  a_{22}x_2 &+& \ldots &+& a_{2n}x_n &=& b_2\\
    \vdots   &&   \vdots   &&         && \vdots  && \vdots\\
    a_{m1}x_1 &+&  a_{m2}x_2 &+& \ldots &+& a_{mn}x_n &=& b_m
  \end{matrix}
\end{equation}
oder kurz $Ax=b$. In dieser allgemeinen Form ist weder garantiert,
dass es eine Lösung gibt (z.B. $A=0$, $b\neq 0$), noch, dass diese
eindeutig ist ($A=0$, $b=0$).

\section{\keyword{Dreiecksmatrizen}}

Eine Matrix $A\in\RR^{n\times n}$ heißt eine \emph{rechte obere
  Dreiecksmatrix}, wenn sie quadratisch ist und $a_{ij}=0$ für $i>j$.
Analog kann man auch die linken unteren Dreiecksmatrizen definieren,
mit $a_{ij}=0$ für $i<j$. In jedem Fall bilden rechte obere und linke
untere Dreiecksmatrizen jeweils Unteralgebren der Matrixalgebra, d.h.,
sie sind abgeschlossen unter Addition und Multiplikation. Die
Schnittmenge dieser Algebren ist wiederum die Algebra der
\emph{\keyword{Diagonalmatrizen}}.

Ist $A$ eine rechte obere Dreiecksmatrix, so hat das Gleichungssystem
die Form
\begin{equation}
  \begin{matrix}
    a_{11}x_1 &+&  a_{12}x_2 &+& \ldots &+& a_{1n}x_n &=& b_1\\
             &&  a_{22}x_2 &+& \ldots &+& a_{2n}x_n &=& b_2\\
             &&            && \ddots && \vdots  && \vdots\\
             &&            &&        && a_{nn}x_n &=& b_n.
  \end{matrix}
\end{equation}
Dieses Gleichungssystem hat genau dann eine Lösung, wenn $A$ regulär
ist, also $\det A=\prod_{i=1}^na_{ii}\neq 0$. Die Lösung kann dann durch
\emph{Rücksubstitution} direkt bestimmt werden:
\begin{equation}
  \label{eq:ruecksubst}
  x_i = \frac{1}{a_{ii}}\left( b_i - \sum_{k=i+1}^n a_{ik}x_k\right)
  \quad\text{für}\; i=n(-1)1.
\end{equation}

Für reguläre \emph{linke untere Dreiecksmatrizen} ergibt sich die
Lösung entsprechend durch \emph{Vorwärtssubstitution}:
\begin{equation}
  \label{eq:vorwsubst}
  x_i = \frac{1}{a_{ii}}\left( b_i - \sum_{k=1}^{i-1} a_{ik}x_k\right)
  \quad\text{für}\; i=1(1)n.
\end{equation}

Für Diagonalmatrizen ist die Situation natürlich einfacher, es gilt
\begin{equation}
  x_i = \frac{1}{a_{ii}} b_i
  \quad\text{für}\; i=1(1)n.
\end{equation}

\begin{sloppypar}
  SciPy stellt für Dreiecksmatrizen spezielle Löserroutinen zur
  Verfügung, \scipy{scipy.linalg.solve_triangular(A, b, lower=False)},
  wobei \argd{lower} angibt, ob  \argd{A} eine linke untere statt rechte
  obere Dreiecksmatrix ist.
\end{sloppypar}

\section{\keyword{Gaußelimination}}

Die Gaußelimination ist ein Verfahren, um eine beliebiges
Gleichungssystem $Ax=b$, mit $A\in\RR^{m\times n}$, auf die
äquivalente Form
{\samepage\vspace{\medskipamount}% für die TikZ-Deko
\begin{equation}
  \label{eq:rform}
  \begin{pmatrix}
    \tikzlabel{rl}\;R\;\tikzlabel{rr} & \tikzlabel{kl}\quad K\quad\tikzlabel{kr} \\
    0 & 0
  \end{pmatrix} x' = b'
\end{equation}
\begin{tikzpicture}[overlay, remember picture]
  \draw[decorate,decoration=brace] ($(rl) + (0,1em)$) --
  node[above] {$k$} ($(rr) + (0,1em)$);
  \draw[decorate,decoration=brace] ($(kl) + (0,1em)$) --
  node[above] {$n-k$} ($(kr) + (0,1em)$);
  \draw[decorate,decoration=brace] ($(rl) + (-1em,-0.1em)$) --
  node[left] {$k$} ($(rl) + (-1em,1em)$);
  \draw[decorate,decoration=brace] ($(rl) + (-1em,-1.4em)$) --
  node[left] {$m-k$} ($(rl) + (-1em,-0.4em)$);
\end{tikzpicture}}

zu bringen, wobei $R\in\RR^{k,k}$ eine reguläre rechte obere Dreiecksmatrix,
$K\in\RR^{k\times n-k}$ beliebig und $x'$ eine Permutation
(Umordnung) von $x$ ist. Dieses Gleichungssystem hat offenbar nur dann
eine Lösung, wenn $b_i'=0$ für $i=k+1(1)m$.

Diese ist im Allgemeinen auch nicht eindeutig, vielmehr können die
freien Variablen $x_\text{K} = (x'_i)_{i=k+1}^{n}$ frei gewählt
werden. Ist $x_\text{R} = (x'_i)_{i=1}^{k}$ der Satz der
verbleibenden Lösungsvariablen, so gilt also
\begin{equation*}
  x_\text{R} = R^{-1}b' - R^{-1}K x_\text{K}.
\end{equation*}
Die Lösungen ergeben sich daraus als
\begin{equation}
  x' =
  \begin{pmatrix}
    R^{-1}b'\\
    0
  \end{pmatrix} +
  \left<
    \begin{pmatrix}
      -R^{-1}K_i\\
      e_i
    \end{pmatrix}
  \right>,
\end{equation}
wobei $K_i$ die $i$-te Spalte von K und
$\left<\right>$ den aufgespannten Vektorraum bezeichnet. Die
Ausdrücke, die $R^{-1}$ enthalten, können durch Rücksubstitution bestimmt
werden.

Um das System $Ax=b$, das wir im Folgenden als $A|b$ zusammenfassen,
auf diese Form zu bringen, stehen folgende Elementaroperationen zur
Verfügung, die offensichtlich die Lösung nicht verändern:
\begin{enumerate}[beginpenalty=10000]
\item Vertauschen zweier Gleichungen (Zeilentausch in $A|b$)
\item Vertauschen zweier Spalten in $x$ und $A$ (Variablenaustausch)
\item Addieren eines Vielfachen einer Zeile zu einer anderen
\item Multiplikation einer Zeile mit einer Konstanten ungleich 0
\end{enumerate}
Die Gaußelimination nutzt nun diese Operationen, um die Matrix
spaltenweise auf die gewünschte Dreiecksform zu bringen. Dazu werden
Vielfache der ersten Zeile von allen anderen abgezogen, so dass die
Gleichung die Form
\begin{equation}
  \left(\begin{array}{lll|c}
      a_{11}^{(0)} & a_{12}^{(0)}\ldots &a_{1n}^{(0)} & b_1^{(0)}\\
      0     & a_{22}^{(1)}\ldots &a_{2n}^{(1)} & b_2^{(1)}\\
      \vdots& \vdots      &\vdots& \vdots\\
      0     & a_{m2}^{(1)}\ldots &a_{mn}^{(1)} & b_m^{(1)}\\
    \end{array}\right) =: A^{(1)} | b^{(1)}
\end{equation}
annimmt, wobei
\begin{equation}
  \begin{array}{rcll}
    a_{ik}^{(1)} &=& a_{ik}^{(0)} -  l_{i}^{(1)}a_{1k}^{(0)}
    & \quad\text{für}\; i=2(1)n, k=1(1)m\\[0.3em]
    b_{i}^{(1)} &=& b_{i}^{(0)} - l_{i}^{(1)} b_{1}^{(0)}
    & \quad\text{für}\; i=2(1)n \\[0.3em]
    a_{1k}^{(1)} &=& a_{1k}^{(0)},\quad b_{1}^{(1)} = b_{1}^{(0)}
    & \quad\text{sonst}
  \end{array}
  \quad\text{mit}\, l_{i}^{(1)} = \frac{a_{i1}^{(0)}}{a_{11}^{(0)}}.
\end{equation}
Mit dem verbleibenden Resttableau wird nun
genauso weiter verfahren:
\begin{equation}
  \label{eq:gausseli}
  \begin{array}{rcll}
    a_{ik}^{(r)} &=& a_{ik}^{(r-1)} -  l_{i}^{(r)}a_{r,k}^{(r-1)}
    & \quad\text{für}\; i=r+1(1)n, k=r(1)m\\[0.3em]
    b_{i}^{(r)} &=& b_{i}^{(r-1)} - l_{i}^{(r)} b_{r}^{(r-1)}
    & \quad\text{für}\; i=r+1(1)n \\[0.3em]
    a_{ik}^{(r)} &=& a_{ik}^{(r-1)},\quad b_{i}^{(r)} = b_{i}^{(r-1)}
    & \quad\text{sonst}
  \end{array}
  \text{mit}\, l_{i}^{(r)} = \frac{a_{ir}^{(r-1)}}{a_{rr}^{(r-1)}}.
\end{equation}
Das Verfahren ist beendet, wenn das Resttableau nur noch eine Zeile hat.

Ist während eines Schrittes $a_{rr}^{(r-1)} = 0$ und
\begin{enumerate}
\item nicht alle $a_{ir}^{(r-1)}=0$, $i=r+1(1)m$. Dann tauscht man
  Zeile $r$ gegen eine Zeile $i$ mit $a_{ir}^{(r-1)}\neq 0$, und fährt
  fort.
\item alle $a_{ir}^{(r-1)}=0$, $i=r(1)m$, aber es gibt ein
  $a_{ik}^{(r-1)}\neq 0$ mit $i\ge r$, $k> r$. Dann vertauscht man zunächst
  Zeile $r$ mit Zeile $i$, tauscht anschließend Spalte $k$ mit
  Spalte $r$, und fährt fort.
\item alle $a_{ik}^{(r-1)}=0$ für $i,k \ge r$. Dann hat
  $A^{(r-1)}|b^{(r-1)}$ die gewünschte Form \eqref{eq:rform} erreicht,
  und das Verfahren terminiert.
\end{enumerate}

Das Element $a_{rr}^{(r-1)}$ heißt auch \emph{Pivotelement}, da es
sozusagen der Dreh- und Angelpunkt des iterativen Verfahrens ist.  In
der Praxis ist es numerisch günstiger, wenn dieses Element möglichst
groß ist. Das lässt sich erreichen, indem wie in den singulären
Fällen verfahren wird, also Zeilen oder Spalten getauscht werden, um
das betragsmäßig maximale $a_{ik}^{(r-1)}$ nach vorne zu
bringen. Folgende Verfahren werden unterschieden
\begin{itemize}
\item \emph{kanonische \keyword{Pivotwahl}}: es wird stets
  $a_{rr}^{(r-1)}$ gewählt und abgebrochen, falls dieses betragsmäßig
  zu klein wird. Diese Verfahren scheitert schon bei einfachen
  Matrizen (z.B. {\tiny $\begin{pmatrix} 0 & 1\\ 1 &
      0\end{pmatrix}$}), und kann daher nur eingesetzt werden, wenn
  die Struktur der Matrix sicherstellt, das $a_{rr}^{(r-1)}$ stets
  hinreichend groß ist.
\item \emph{Spaltenpivotwahl}: es wird wie oben im 1. Fall nur in der
  Spalte maximiert, d.h.\ wir wählen als  Pivotelement
  \begin{equation}
    i_0 = \text{argmax}_{i\ge r} \lvert a_{ir}^{(r-1)}\rvert
  \end{equation}
  und tauschen Zeilen $i_0$ und $r$; die Variablenreihenfolge bleibt
  unverändert. Ist die Matrix $A$ quadratisch, bricht das Verfahren
  genau dann ab, wenn $A$ singulär ist.
\item \emph{Totalpivotwahl}: wie oben im 2. Fall wird stets das
  betragsmäßig maximale
  Matrixelement im gesamten Resttableau gesucht, also
  \begin{equation}
    i_0,k_0 = \text{argmax}_{i,k\ge r} \lvert a_{ik}^{(r-1)}\rvert.
  \end{equation}
  Dann vertauscht man zunächst Zeile $r$ mit Zeile $i_0$, und tauscht
  anschließend Spalte $k_0$ mit Spalte $r$, wobei man sich noch die
  Permutation der Variablen geeignet merken muss, zum Beispiel als
  Vektor von Indizes.
\end{itemize}

Unabhängig von der Pivotwahl benötigt die Gaußelimination bei
quadratischen Matrizen im wesentlichen $\O(n^3)$
Fließkommaoperationen. Das ist relativ langsam, daher werden wir
später bessere approximative Verfahren kennenlernen. Für Matrizen
bestimmter Struktur, zum Beispiel Bandmatrizen, ist die
Gaußelimination aber gut geeignet. NumPy bzw. SciPy stellen daher auch
keine Gaußelimination direkt zur Verfügung.
\scipy{scipy.linalg.solve(A, b)} ist ein Löser für Gleichungssysteme
$Ax=b$, der immerhin auf der LU-Zerlegung durch Gaußelimination
basiert. Dieser Löser setzt allerdings voraus, dass die Matrix nicht
singulär ist, also eindeutig lösbar.
 
\section{\keyword{Matrixinversion}}

Ist $A\in\RR^{n\times n}$ regulär, so liefert die Rücksubstitution
implizit die Inverse von $A$, da für beliebige $b$ das
Gleichungssystem $Ax=b$ gelöst werden kann. Allerdings muss das für
jedes $b$ von neuem geschehen. Alternativ kann mit Hilfe der
Gaußelimination auch die Inverse von $A$ bestimmt werden. Dazu wird
das Tableau $A|I$ in das Tableau $I|A^{-1}$ transformiert, wobei $I$
die $n\times n$-Einheitsmatrix bezeichnet. Die Vorgehensweise
entspricht zunächst der Gaußelimination mit
Spaltenpivotwahl. Allerdings werden nicht nur die Elemente unterhalb
der Diagonalen, sondern auch oberhalb eliminiert. Zusätzlich wird die
Pivotzeile noch mit $1/a_{ii}^{(i-1)}$ multipliziert, so dass das $A$
schrittweise die Form
\begin{equation}
  \left(\begin{array}{cccc}
      1     & 0      & a_{12}^{(2)}\ldots &a_{1n}^{(2)} \\
      0     & 1      & a_{22}^{(2)}\ldots &a_{2n}^{(2)} \\
      \vdots& 0      & a_{32}^{(2)}\ldots &a_{3n}^{(2)} \\
      \vdots& \vdots & \vdots             & \vdots      \\
      0     & 0      & a_{n2}^{(2)}\ldots &a_{nn}^{(2)} \\
    \end{array}\right)
\end{equation}
annimmt. Das Verfahren ist allerdings numerisch nicht sehr stabil, und
generell sollte die explizite Berechnung der Inversen wann immer
möglich vermieden werden. SciPy stellt die Matrixinversion als
Funktion \scipy{scipy.linalg.inv(A)} zur Verfügung.

Eine Ausnahme bilden Matrizen der Form $I + A$ mit $\lVert A \rVert =
\max \lVert Ax\rVert / \lVert x \rVert < 1$. Dann ist
\begin{equation}
  (I+A)^{-1} = I - A + A^2 - A^3 + \cdots
\end{equation}
eine gut konvergierende Näherung der Inversen.

\section{\keyword{LU-Zerlegung}}
\index{LR-Zerlegung}

Eine weitere Anwendung der Gaußelimination ist die LU-Zerlegung von
bestimmten quadratischen Matrizen. Dabei wird eine Matrix
$A\in\RR^{n\times n}$ so in eine ein linke untere Dreiecksmatrix $L$
und eine rechte obere Dreiecksmatrix $U$ zerlegt, dass $A=L\cdot U$.
Um die LU-Zerlegung eindeutig zu machen, vereinbart man üblicherweise,
dass $l_{ii}=1$ für $i=1(1)n$. Das $U$ steht übrigens für das
englische "`upper right"' und $L$ für "`lower left"'. Im Deutschen
findet sich vereinzelt noch der Begriff LR-Zerlegung, wobei hier L für
eine linke untere und R für eine rechte obere Matrix steht.

Ist eine solche Zerlegung einmal gefunden, lässt sich das
Gleichungssystem $Ax=b$ für beliebige $b$ effizient durch Vorwärts-
und Rücksubstitution lösen:
\begin{equation}
  L y = b,\, U x = y \quad\implies\quad Ax = L\,U x = L y = b.
\end{equation}
Zunächst wird also $y$ durch Vorwärtssubstitution berechnet,
anschließend $x$ durch Rückwärtssubstitution. Die Inverse lässt
sich so auch bestimmen:
\begin{equation}
  L y_i = e_i,\, U x_i = y_i\quad\text{für}\; i=1(1)n
  \quad\implies\quad  A^{-1} = \left(x_1,
    \ldots, x_n\right).
\end{equation}
Die Determinante von $A=L\cdot U$ ist ebenfalls einfach zu bestimmen:
\begin{equation}
  \det A = \det L\,\det U = \prod_{i=1}^n u_{ii}
\end{equation}

Um die LU-Zerlegung zu berechnen, nutzen wir wieder die
Gaußelimination.  Kann bei $A\in\RR^{n\times n}$ die Gaußelimination
in kanonischer Pivotwahl durchgeführt werden, so ist die LU-Zerlegung
von $A$ durch $U=A^{(n-1)}$, also die finale, auf rechte obere
Dreiecksform transformierte Matrix, und durch die Matrix
\begin{equation}
  L =
  \begin{pmatrix}
    1       &          &         &          & 0 \\
    l_1^{(0)} & 1       &         &           &  \\
    l_2^{(0)} & l_2^{(1)} & 1       &          &  \\
    \vdots  &          & \ddots & \ddots    & \\
    l_n^{(0)} & \ldots  & \ldots  & l_n^{(n-1)} & 1
  \end{pmatrix}
\end{equation}
der Updatekoeffizienten aus \eqref{eq:gausseli} gegeben.

Wie bereits gesagt, ist die Voraussetzung, dass die Gaußelimination
mit kanonischer Pivotwahl durchgeführt werden kann, sehr restriktiv, und
schließt selbst einfache Matrizen wie {\tiny $\begin{pmatrix} 0 & 1\\
    1 & 0\end{pmatrix}$} aus. Wie man sich leicht überlegt, besitzt
diese Matrix allerdings keine LU-Zerlegung.

Für manche Anwendungen ist es günstiger, wenn $L$ und $U$ normiert
sind. Dann benutzt man die \keyword{LDU-Zerlegung} $A=LDU$, mit $L$
linker unterer Dreiecksmatrix, $D$ Diagonalmatrix und $U$ rechter
oberer Dreiecksmatrix. Jetzt müssen $l_{ii}=1$ und $r_{ii}=1$
sein. Die LDU-Zerlegung ergibt sich aus der LU-Zerlegung durch $d_{ii}
= u_{ii}$ und $u'_{ik}=u_{ik}/u_{ii}$.

In SciPy ist die LU-Zerlegung als
\scipy{scipy.linalg.lu_factor(A)} oder \scipy{scipy.linalg.lu(A)} (zur
Zerlegung der Matrix A) und \scipy{scipy.linalg.lu_solve((lu,piv), b)}
(zum Lösen des LGS) implementiert.

\section{Cholesky-Zerlegung}
\index{Cholesky>-Zerlegung}

Wir betrachten im Folgenden nur symmetrische, positiv definite
Matrizen, wie sie gerade in der Physik oft vorkommen. Auch in der
Optimierung spielen diese eine wichtige Rolle. Sei $A=LDU$ eine
LDU-Zerlegung einer symmetrischen Matrix, dann gilt
\begin{equation}
  LDU = A = A^T = (LDU)^T = U^T D L^T,
\end{equation}
wobei $^T$ wie eingangs erwähnt die Transponierte von $A$ bezeichnet.
Da die LDU-Zerlegung aber eindeutig ist und $U^T$ eine normierte,
linke untere Dreiecksmatrix und $L^T$ eine normierte, rechte obere
Dreiecksmatrix, so gilt $U=U^T$, und damit
\begin{equation}
  A = U^TDU = R^TR \quad\text{mit}\; R=\text{diag}(\sqrt{d_{ii}})U.
\end{equation}
Dies ist die Cholesky-Zerlegung. Anstatt die Gaußelimination
durchzuführen, lässt sich die Zerlegung aber auch direkt mit Hilfe des
\emph{Cholesky-Verfahrens}\index{Cholesky>-Verfahren} bestimmen: Sei
$A=R^TR$ eine Cholesky-Zerlegung.
Da $R$ unterhalb der Diagonalen nur 0 enthält, gilt
\begin{equation}
  a_{ik} = \sum_{l=1}^{i} r_{li}r_{lk} \quad\text{für}\;
  i=1(1)n,\,k=1(1)n.
\end{equation}
Daraus lässt sich die erste Zeile von $R$ direkt ablesen:
\begin{equation}
r_{11} = \sqrt{a_{11}}\quad\text{und}\;
r_{1k} = \frac{a_{1k}}{r_{11}} \quad\text{für}\;k=2(1)n.
\end{equation}
Die nächsten Zeilen lassen sich analog bestimmen, da für jedes $i$
\begin{equation}
  a_{ii} = \sum_{l=1}^{i} r_{li}^2\quad\implies\;
  r_{ii} = \sqrt{a_{ii} - \sum_{l=1}^{i-1} r_{li}^2}.
\end{equation}
Für die restlichen Elemente der Zeile gilt
\begin{equation}
  r_{ik} = \frac{1}{r_{ii}}\left(a_{ik} - \sum_{l=1}^{i-1}r_{li}r_{lk}\right) \quad\text{für}\;k=i+1(1)n
\end{equation}
Das Cholesky-Verfahren ist wie die Gaußelimination von der Ordnung
$\O(n^3)$, braucht aber nur halb so viele Operationen. In SciPy ist die
Cholesky-Zerlegung als \scipy{scipy.linalg.cholesky(A)} implementiert.

\section{\keyword{Bandmatrizen}}

Im Folgenden werden wir oft mit $k$-Bandmatrizen zu tun haben, also
Matrizen, bei denen nur die Diagonale und einige Nebendiagonalen
besetzt sind. Diagonalmatrizen sind also 1-Bandmatrizen, eine
\emph{Dreibandmatrix}\index{Dreibandmatrizen} hat die Form
\begin{equation}
  \begin{pmatrix}
    d_1 & t_1    &     & & 0 \\
    b_1 & d_2    & t_2 \\
    {}  & \ddots & \ddots & \ddots \\
    {}  &        & b_{n-2} & d_{n-1} & t_{n-1}\\
    0   &        &        & b_{n-1} & d_n
  \end{pmatrix}.
\end{equation}
Für Matrizen dieser Form ist die Gaußelimination mit kanonischer
Pivotwahl sehr effizient, da pro Iteration jeweils nur die erste Zeile
des Resttableaus verändert werden muss. Dadurch ist der Rechenaufwand
nur noch linear in der Matrixgröße bzw. Länge der Bänder. Die
Dreiecksmatrizen $L$ und $U$ der LU-Zerlegung sind zusätzlich
(Drei-) Bandmatrizen, wobei $L$ nur auf der Haupt und der unteren
Nebendiagonalen von Null verschiedene Einträge hat, $U$ nur auf der
Diagonalen und der Nebendiagonalen oberhalb.

SciPy stellt für Bandmatrizen ebenfalls spezielle Löserroutinen zur
Verfügung, \scipy{scipy.linalg.solve_banded((l,u), A, b)},
wobei \argd{l} und \argd{u} die Anzahl der Nebendiagonalen oberhalb
und unterhalb angeben, und \argd{A} die Matrix in Bandform
angibt.

%%% Local Variables: 
%%% mode: latex
%%% TeX-master: "padc"
%%% TeX-PDF-mode: t
%%% End: 
